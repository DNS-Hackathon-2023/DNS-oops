\documentclass[Screen16to9,17pt]{foils}
%\documentclass[16pt,landscape,a4paper,footrule]{foils}
\usepackage{hackathon-oops-slides}

% SPDX-FileCopyrightText: 2006-2021, Henrik Kramselund <hlk@kramse.org>
% SPDX-License-Identifier: BSD 3-Clause "New" or "Revised" License
% This is an example presentation

\begin{document}

%\rm
\selectlanguage{english}

\mytitlepage
{DNS OOPS\\\small Notify the BGP daemons}

\LogoOn

\slide{Goals for the project}
\vskip 2 cm

%{\hlkbig En 3 dages workshop, hvor du lærer at angribe dit netværk!}

\begin{quote}

   This document seeks to specify {\bf a method for name servers} to {\bf signal
   programs outside} of the {\bf name server software}, and which are not
   necessarily involved with the DNS protocol, {\bf about conditions} that can
   arise {\bf within the name server}.  These {\bf signals} can be used to {\bf invoke
   actions} in areas that {\bf help provide the DNS service}, such as {\bf routing}.

\end{quote}

Main URL:\\
\url{https://datatracker.ietf.org/doc/draft-grubto-dnsop-dns-out-of-protocol-signalling/}
\begin{list1}
\item
\end{list1}

\slide{The goal}

\hlkimage{14cm}{images/DNS Signalling.drawio.png}

Automated signalling when things happen in the DNS server ...

\slide{Small configuration file}

\begin{minted}[fontsize=\scriptsize]{JSON}
{
    "_comment": "Some information should be templated",
    "zones": {
        "oops.example.com.": {
            "last-octet": 4,
            "generate-dsc": true,
            "generate-pcap": false
        }
    }
}
\end{minted}


\slide{Technologies used }


\begin{list2}
\item DNS server software: Knot DNS, BIND, NSD
\item BGP software: Bird, ExaBGP
\item Programming languages: Python 3, Shell scripting
\item Notification: shell scripting, D-bus to Knot DNS
\end{list2}




\slide{Example: Use the Libraries}

First Python script uses various features:
\begin{minted}[fontsize=\small]{python}
#!/usr/bin/env python3

import sys
import dbus   // D-bus is implemented in Knot DNS
import dbus.mainloop.glib
import signal
import requests
import json
import libknot.control
from gi.repository import GLib

api_url = None
node_data_file = '/root/node_data.json'
zones = {}
\end{minted}


\slide{Example: Event loop and D-bus}

When zone is updated, do something:
\begin{minted}[fontsize=\small]{python}
def updated(*args, **kwargs):
    """
    Event handler for Glib.MainLoop
    Also explicitly called at script startup to determine if any event handling needs to be done
    """
    (zone, serial) = args
    print("Zone %s updated, SOA serial %d" % (zone, serial))
    zones[zone] = serial
    if None not in zones.values():
        # Build a list of all zones with serials
        all_zone_info = ""
        for key in zones:
            all_zone_info += f'{key}:{zones[key]};'
...
\end{minted}

This is an example, can be run inside/outside of the server, in cloud systems etc.


\slide{Improvements and next steps}



\begin{list2}
\item Create dockerfiles for packing this in an easy way
\item Add some bridging between Notify and D-bus
\item Add D-bus features to NSD
\end{list2}




\end{document}

\slide{Example preformatted text}

\begin{alltt}
\footnotesize # {\bfseries tcpdump -i en0 host 10.20.20.129 or host 10.0.0.11}
tcpdump: listening on en0
23:23:30.426342 10.0.0.200.33849 > router.33435: udp 12 {\bf [ttl 1]}
23:23:30.426742 safri > 10.0.0.200: {\bf icmp: time exceeded in-transit}
23:23:30.436069 10.0.0.200.33849 > router.33436: udp 12 {\bf [ttl 1]}
23:23:30.436357 safri > 10.0.0.200: {\bf icmp: time exceeded in-transit}
23:23:30.437117 10.0.0.200.33849 > router.33437: udp 12 {\bf [ttl 1]}
23:23:30.437383 safri > 10.0.0.200: {\bf icmp: time exceeded in-transit}
23:23:30.437574 10.0.0.200.33849 > router.33438: udp 12
23:23:30.438946 router > 10.0.0.200: icmp: router {\bf udp port 33438 unreachable}
23:23:30.451319 10.0.0.200.33849 > router.33439: udp 12
23:23:30.452569 router > 10.0.0.200: icmp: router {\bf udp port 33439 unreachable}
23:23:30.452813 10.0.0.200.33849 > router.33440: udp 12
23:23:30.454023 router > 10.0.0.200: icmp: router {\bf udp port 33440 unreachable}
23:23:31.379102 10.0.0.200.49214 > safri.domain:  6646+ PTR?
200.0.0.10.in-addr.arpa. (41)
23:23:31.380410 safri.domain > 10.0.0.200.49214:  6646 NXDomain* 0/1/0 (93)
14 packets received by filter
0 packets dropped by kernel
\end{alltt}


\slide{Example pictures in slide - TCP three-way handshake}

\hlkimage{5cm}{images/tcp-three-way.pdf}

\begin{list2}
\item {\bfseries TCP SYN half-open} scans
\item Tidligere loggede systemer kun når der var etableret en fuld TCP
  forbindelse\\
  -- dette kan/kunne udnyttes til \emph{stealth}-scans
\item Hvis en maskine modtager mange SYN pakker kan dette fylde
  tabellen over connections op -- og derved afholde nye forbindelser
  fra at blive oprette -- {\bfseries SYN-flooding}
\end{list2}



\slide{Example: Picture and code}

\hlkimage{18cm}{buffer-overflow-1.pdf}

\begin{minted}[fontsize=\small]{C}
main(int argc, char **argv)
{      char buf[200];
        strcpy(buf, argv[1]);
        printf("%s\textbackslash{}n",buf);
}
\end{minted}



\slide{VXLAN injection}

\hlkimage{19cm}{vxlan-basic-injection.png}

I tested using my pentest server in one AS, sending across an internet exchange into a production network, towards Arista testing devices - no problems, it's just routed layer 3 IP+UDP packets

\slide{Example attacks, What is possible VXLAN Header}

\begin{alltt}\footnotesize
+-+-+-+-+-+-+-+-+-+-+-+-+-+-+-+-+-+-+-+-+-+-+-+-+-+-+-+-+-+-+-+-+
|R|R|R|R|I|R|R|R|            Reserved                           |
+-+-+-+-+-+-+-+-+-+-+-+-+-+-+-+-+-+-+-+-+-+-+-+-+-+-+-+-+-+-+-+-+
|                VXLAN Network Identifier (VNI) |   Reserved    |
+-+-+-+-+-+-+-+-+-+-+-+-+-+-+-+-+-+-+-+-+-+-+-+-+-+-+-+-+-+-+-+-+
Inner Ethernet Header:
+-+-+-+-+-+-+-+-+-+-+-+-+-+-+-+-+-+-+-+-+-+-+-+-+-+-+-+-+-+-+-+-+
|             Inner Destination MAC Address                     |
+-+-+-+-+-+-+-+-+-+-+-+-+-+-+-+-+-+-+-+-+-+-+-+-+-+-+-+-+-+-+-+-+
| Inner Destination MAC Address | Inner Source MAC Address      |
+-+-+-+-+-+-+-+-+-+-+-+-+-+-+-+-+-+-+-+-+-+-+-+-+-+-+-+-+-+-+-+-+
|                Inner Source MAC Address                       |
+-+-+-+-+-+-+-+-+-+-+-+-+-+-+-+-+-+-+-+-+-+-+-+-+-+-+-+-+-+-+-+-+
|OptnlEthtype = C-Tag 802.1Q    | Inner.VLAN Tag Information    |
+-+-+-+-+-+-+-+-+-+-+-+-+-+-+-+-+-+-+-+-+-+-+-+-+-+-+-+-+-+-+-+-+
\end{alltt}

\begin{list2}
\item Above protocol header is copied from RFC text document, and in \verb+alltt+ environment
\end{list2}



\slide{Example: Snippets of code with minted Scapy}

First create VXLAN header and inside packet
\begin{minted}[fontsize=\small]{python}
vxlanport=4789     # RFC 7384 port 4789, Linux kernel default 8472
vni=37             # Usually VNI == destination VLAN
vxlan=Ether(dst=routermac)/IP(src=vtepsrc,dst=vtepdst)/
   UDP(sport=vxlanport,dport=vxlanport)/VXLAN(vni=vni,flags="Instance")
broadcastmac="ff:ff:ff:ff:ff:ff"
randommac="00:51:52:01:02:03"
attacker="185.27.115.666"
destination="10.0.0.10"
# port is the one we want to contact inside the firewall
insideport=53
testport=54040
packet=vxlan/Ether(dst=broadcastmac,src=randommac)/IP(src=attacker,
    dst=destination)/UDP(sport=testport,dport=insideport)/
    DNS(rd=1,id=0xdead,qd=DNSQR(qname="www.wikipedia.org"))
\end{minted}

{\footnotesize Fun fact, Unbound on OpenBSD reply to DNS requests received in Ethernet packets with broadcast destination and IP destination being the IP of the server}

\myquestionspage

\end{document}
